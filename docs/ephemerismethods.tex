\mychapter{The different calculation methods}


\mysection{Variations Séculaires des Orbites Planétaires (VSOP)}

\mysubsection{vsop87}

The error bound of vsop87 calculations as seen from section 3 of \cite{vsop87}\footnote{Strangely, this doesn't seem to correspond exactly to the values given in \texttt{vsop87.doc} as found on \url{ftp://ftp.imcce.fr/pub/ephem/planets/vsop87/vsop87.doc}.} are, converted in seconds:

\begin{table}[h]
\centering
\sisetup{table-format=1.1, table-figures-exponent = 2}
\begin{tabular}{|l|S|}
\hline
\textbf{Celestial body} & \multicolumn{1}{c|}{\textbf{Maximum error}} \\\hline
\textbf{Earth (Sun)} & 5e-3 \\\hline % 5.1e-3 s = 2.5e-8 rad in vsop87.doc
\textbf{Mercury} & 1e-3 \\\hline % 1.2e-3s = 0.6e-8 rad
\textbf{Venus} & 6e-3 \\\hline % 5.1e-3s = 2.5e-8 rad
\textbf{Mars} & 2.3e-2 \\\hline % 2e-2s = 10e-8 rad
\textbf{Jupiter} & 2e-2 \\\hline % 7.2e-2s = 35e-8 rad
\textbf{Saturn} & 0.1 \\\hline % 1.4e-1s = 70e-8 rad
\end{tabular}
\caption{Minimum trueness of vsop87}
\label{table:vsopprecision}
\end{table}

The guaranteed maximal error is \ang{;;1} in the interval
\begin{itemize}
\item $[0,4000]$ for Jupiter and Saturn
\item $[-2000,6000]$ for the other bodies\footnote{except Neptune and Uranus, but these aren't used in calendars.}
\end{itemize}

\mysubsection{vsop2013}

\mysection{ELP/MPP02 and LEA-406}

\mysection{JPL Ephemeris}
