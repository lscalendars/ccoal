\mychapter{What precision for calendrical calculations?}

\mysection{Scales used in lunisolar calendars}

\mysubsection{The tibetan calendar}

\mysubsubsection{Angular scales}

A typical Tibetan almanach contains angle data written in the following form $x;y;z$ where $x$ is the lunar mansion (27\textsuperscript{th} of circle), $y$ is in nâdis (60\textsuperscript{th} of lunar mansion) and $z$ in pâlas (60\textsuperscript{th} of nâdi). It is almost the same as Western angle notation, but dividing the circle first in 27, not in 360. So a first approach would say that computations must be accurate enough to get these numbers right. In order to get these right, the precision must be at least $0.5 pâla = 6.67 s$\footnote{$1 pâla$ being a $27\times60\times60^{th}$ of circle and $1s$ being a $360\times60\times60^{th}$ of circle, we have $1 pâla = \frac{360}{27} s = 13.33 s$}.

Internally, calculations are made with more precision, adding two more sets of digits, $x;y;z$ become $x;y;z;z_2;z_3$ where $z_2$ is in shvâsa (6\textsuperscript{th} of pâla) and $z_3$ in bhâga, a fraction of shvâsa that varies according to calculation type. They are mostly:
\begin{itemize}
\item 707, used for Sun and Moon, corresponding to a precision of \num{3.14e-3}\,s
\item 149209, used for planets, corresponding to a precision of \num{1.49e-5}\,s
\end{itemize}

\mysubsubsection{Time scale}

Times is expressed almost in the same way as angles, in the form $x;y;z$ where $x$ is the day of the week (0 is saturday), $y$ is in nâdis (60\textsuperscript{th} of day, 24 minutes) and $z$ in pâlas (60th of nâdi, 24s).

Subunits used internally are shvâsa (6\textsuperscript{th} of pâla, 4s) and the same kind of fraction of shvâsa as angle scales.

\mysubsection{Other calendars}
