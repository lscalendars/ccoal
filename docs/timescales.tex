\mychapter{Timescales}

\mysection{Introduction}

It might first seem strange to talk about timescales, the obvious belief is that there is only one timescale and we seasure events on it. Sadly, this is not the case and some errors might occur due to this belief. In this chapter, we will briefly describe the different timescales, their relation, and their impact on calculations.

\mysubsection{Impact of the error on time}

As all our numerical data is in the second of arc unit, we need to first make some calculations that will allow conversion from an error on time (expressed in seconds) in an error of angle, expressed in seconds of arc. Of course, this conversion depends on too many things to get even an approximation of the function giving this conversion. But what's interesting for us is the maximum conversion rate, meaning the conversion rate in the case where it will induce the maximal anglar error. 

As we are interested in majoring this conversion rate, we will also use many pessimistic approximations. The data we'll use for celestial bodies correspond to the following case:

\begin{itemize}
\item the celestial bodies are at their closest position from the Earth
\item the celestial bodies are at their maximum speed (even the Earth)
\item the observation takes place at the top of Mt Everest, on the Ecliptic, when the observer is closest to the celestial body
\end{itemize}

As we consider short times, we'll make the following approximations:
\begin{itemize}
\item the planet goes in a linear direction
\item due to the rotation of the Earth on itself, the observer goes in the opposite linear direction
\item due to the rotation of the Earth around the Sun, the observer goes in the same linear direction as the one just before
\end{itemize}

A few notations:
\begin{itemize}
\item $r_{min}$ is the distance between the observer and the center of the planet, which we'll consider constant because of the short times
\item $v_{p}$ is the maximum linear speed of the planet
\item $v_{obs}=v_{E}+v_{S}$ is the maximum linear speed of the observer, the sum of the maximum linear speed of the observer due to the rotation of the Earth on itself and around the Sun. For the Sun and the Moon, we only have $v_{obs}=v_{E}$.
\end{itemize}

We'll express distances in kilometers, times in seconds and speeds in kilometers by second.

A first thing can be to determine $v_{obs}=v_E+v_S$. If we consider the observer as we defined it, and say Earth is round, in a 24h day, he'll travel $2\pi\times r_{E}$ where $r_{E}=R_\oplus+alt._{obs}=6375.9km$ is the distance between him and the geocenter. We have thus 

\begin{equation}
v_E = \frac{2\pi r_E}{86400} = 0.46km/s
\end{equation}

Table~\cite{table:planetvalues} gives $v_S < 30.29 km/s$. For planets, we can thus neglect $v_E$ and major $v_{obs}$:

\begin{equation}
v_{obs} < 30.29 km/s
\end{equation}

Now, the maximum error angle by second is the difference between the angle and the angle one second later. It has its maximum when the first angle is 0, so we just need to calculate the angle after one second. For planets, this angle corresponds to Fig~\ref{maxangularerror}.

\begin{figure}
\centering
\begin{tikzpicture}[scale=2]
\draw [<-] (-2,0) node[anchor=north] {$P_{obs}(t_1)$} -- node[anchor=north] {$\vec{v}_{obs}$} (0,0) node[anchor=north] {$P_{obs}(t_0)$} -- %node[northwest=2mm] {$r_{min}$}
(0,2) node[anchor=south] {$P_{cb}(t_0)$};
\node [anchor=center] at (-0.25,1.35) {$r_{min}$};
\draw [->] (0,2) -- node[anchor=south] {$\vec{v}_{cb}$} (2,2) node[anchor=south] {$P_{cb}(t_1)$};
\draw [gray] (2,2) -- (-2,0);
\draw (-1.3,0) arc (0:26.56:0.707);
\node [anchor=center] at (-1,0.2) {$\Theta_{max}$};
\end{tikzpicture}
\caption{Maximum angular error for short times}\label{maxangularerror}
\end{figure}

This figure can be easily simplified in a simple triangle and we can deduce $\Theta_{max}$ for planets:

\begin{equation}
\Theta_{max}=atan(\frac{v_{obs}-v_{cb}}{r_{min}})
\end{equation}

For Sun and Moon, the reference is not moving, so we can simply take the same value with $v_{cb}=0$. Table~\cite{table:planetvalues} gives $r_{min}$ and $v_{cb_{max}}$, we can thus deduce Table~\cite{table:thetamaxtime}. Note that the values for the planets are over-pessimistic and could be refined by taking the account that all planets rotate in the same direction. We don't need to do this because the biggest error, which we'll take as reference, is on the Moon which has a quite accurate value.

\begin{table}
\centering
\begin{tabular}{|l|S[table-format=1.1,table-figures-exponent=1]|}
\hline
\multicolumn{1}{|c|}{\textbf{Celestial body}} & \multicolumn{1}{c|}{\textbf{$\Theta_{max}$ (s)}} \\\hline
Sun & 4.3e-2\\\hline
Moon & 6.2e-1\\\hline
Mercury & 2.4e-1\\\hline
Mars & 2.1e-1\\\hline
Venus & 3.6e-1\\\hline
Saturn & 6.9e-3\\\hline
Jupiter & 1.1e-1\\\hline
\end{tabular}
\caption{Maximum angular error for a time error of 1s}
\label{table:thetamaxtime}
\end{table}

\mysubsection{Impact of angular error on time}

In this section we study the impact of an angular error on a time determination. The typical case is the computation of the time of full moon (or any moon phase). This is done by computing the time at which the angle of the sun and the angle of the moon are separated by $\pi$. But both angle computations will contain errors, what we will do here is to bind the effect of this error on the computed time.

We won't consider planets here, as they never intervene in this kind of computations, but we'll consider the phase of the moon. This also simplifies the calculations as we can consider $v_{obs}=0$.

For a small angular error $\Theta$, let's consider $d_{cb}$ the distance crossed by the celestial body for to this angle. If we note $\Delta t$ the time error corresponding to this angle, we have $d_{cb}=v_{cb}\times \Delta t$. Let's consider Fig~\ref{distanceerror}.

\begin{figure}
\centering 
\begin{tikzpicture}[scale=2]
\draw (0,0) node[anchor=north] {$P_{obs}$} -- (0,2) node[anchor=south] {$P_{cb}(t_0)$};
\node [anchor=center] at (-0.25,1) {$r$};
\draw (0,2) -- node[anchor=south] {$d_{cb}$} (2,2) node[anchor=south] {$P_{cb}(t_1)$};
\draw (2,2) -- (0,0);
\draw (0.5,0.5) arc (45:90:0.707);
\node [anchor=center] at (0.35,0.8) {$\Theta$};
\end{tikzpicture}
\caption{Distance error for angular error $\Theta$}\label{distanceerror}
\end{figure}

We have the relationship

\begin{equation}
tan(\Theta) = \frac{d_{cb}}{r}=\frac{v_{cb}\times \Delta t}{r}
\end{equation}

From which we deduce:

\begin{equation}
\Delta t = \frac{atan(\Theta)\times r}{v_{cb}} < \frac{atan(\Theta)\times r_{max}}{v_{cb_{min}}}
\end{equation}

So, considering that the error for the phase of the Moon is the error for the Moon plus the error for the Sun, we can compute Table~\ref{table:maxtimeerror}.

\begin{table}
\centering
\begin{tabular}{|l|S[table-format=2.2,table-figures-exponent=0]|}
\hline
\multicolumn{1}{|c|}{\textbf{Angular error for}} & \multicolumn{1}{c|}{\textbf{$\Delta t_{max}$} (s)} \\\hline
Sun & 25.15 \\\hline
Moon & 2.06 \\\hline
Phase of the Moon & 27.21 \\\hline
\end{tabular}
\caption{Maximum error on time for an angular error $\Theta$ = 1s} % 1s=0.0000048481368 rad, atan(1s)=0.0000048481368
\label{table:maxtimeerror}
\end{table}

\mysection{Astronomical timescales}

\mysubsection{Atomic time (TAI)}

\mysubsection{Earth Rotation Time}

\mysubsection{Dynamical Time}

\mysubsubsection{Based on TAI}

\mysubsubsection{Truely relativistic time}

\mysection{Time used in calendrical calculations}

\mysubsection{Mean Sun?}

\mysection{Conversions}
